\documentclass{amsart}

\numberwithin{equation}{section}
\newtheorem{thm}{Theorem}
\newcommand\cc[1]{\texttt{\symbol{92}#1}}

\begin{document}

\begin{align*}
    \sum_{\substack{i < n\\i \text{ even}}} x_i^2\\
    \sum_{\begin{subarray}{l}i < n\\i \text{ even}\end{subarray}} x_i^2
\end{align*}

\begin{table}[h]
    \renewcommand\arraystretch{1.75}
    \begin{center}
        \begin{tabular}{lcc}
            \hline
            Type &Inline &Displayed\\\hline
            \verb+\int_a^b+ &$\int_a^b$ &$\displaystyle \int_a^b$\\
            \verb+\oint_a^b+ &$\oint_a^b$ &$\displaystyle \oint_a^b$\\
            \verb+\iint_a^b+ &$\iint_a^b$ &$\displaystyle \iint_a^b$\\
            \verb+\iiint_a^b+ &$\iiint_a^b$ &$\displaystyle \iiint_a^b$\\
            \verb+\iiiint_a^b+ &$\iiiint_a^b$ &$\displaystyle \iiiint_a^b$\\
            \verb+\idotsint_a^b+ &$\idotsint_a^b$ &$\displaystyle \idotsint_a^b$\\
            \verb+\prod{i=1}^n+ &$\prod{i=1}^n$ &$\displaystyle \prod{i=1}^n$\\
            \verb+\coprod{i=1}^n+ &$\coprod{i=1}^n$ &$\displaystyle \coprod{i=1}^n$\\
            \verb+\bigcap_{i=1}^n+ &$\bigcap_{i=1}^n$ &$\displaystyle \bigcap_{i=1}^n$\\
            \verb+\bigcup_{i=1}^n+ &$\bigcup_{i=1}^n$ &$\displaystyle \bigcup_{i=1}^n$\\
            \verb+\bigwedge_{i=1}^n+ &$\bigwedge_{i=1}^n$ &$\displaystyle \bigwedge_{i=1}^n$\\
            \verb+\bigvee_{i=1}^n+ &$\bigvee_{i=1}^n$ &$\displaystyle \bigvee_{i=1}^n$\\
            \verb+\bigsqcup_{i=1}^n+ &$\bigsqcup_{i=1}^n$ &$\displaystyle \bigsqcup_{i=1}^n$\\
            \verb+\biguplus_{i=1}^n+ &$\biguplus_{i=1}^n$ &$\displaystyle \biguplus_{i=1}^n$\\
            \verb+\bigotimes_{i=1}^n+ &$\bigotimes_{i=1}^n$ &$\displaystyle \bigotimes_{i=1}^n$\\
            \verb+\bigoplus_{i=1}^n+ &$\bigoplus_{i=1}^n$ &$\displaystyle \bigoplus_{i=1}^n$\\
            \verb+\bigodot_{i=1}^n+ &$\bigodot_{i=1}^n$ &$\displaystyle \bigodot_{i=1}^n$\\
            \verb+\sum_{i=1}^n+ &$\sum_{i=1}^n$ &$\displaystyle \sum_{i=1}^n$\\
            \hline
        \end{tabular}
        \caption{Large operators.}
    \end{center}
\end{table}
\[
    \frac{z^d - z_0^d}{z - z_0} = \sum_{k=1}^d z_0^{k-1}z^{d-k} \text{\quad and \quad}
    (T^n)'(x_0) = \prod_{k=0}^{n-1} T'(x_k)
\]

Inline $\bigsqcup\limits_\mathfrak m X = a$

\begin{align*}
    \sum\nolimits_{i=1}^n x_i^2 \quad \sum_{i=1}^n x_i^2\\
    \bigsqcup\nolimits_\mathfrak f X = a\\
    \mathfrak{Castlevania}
\end{align*}

\begin{table}[h]
    \begin{center}
        \begin{tabular}{ll}
            \hline
            Type &Typeset\\\hline
            \verb+$a \equiv v \mod\theta$+ &$a \equiv v \mod\theta$\\
            \verb+$a \bmod b$+ &$a \bmod b$\\
            \verb+$a \equiv v \pmod\theta$+ &$a \equiv v \pmod\theta$\\
            \verb+$a \equiv v \pod\theta$+ &$a \equiv v \pod\theta$\\
            \hline
        \end{tabular}
        \caption{Congruences.}
    \end{center}
\end{table}

\begin{align*}
    \injlim\nolimits_{x \to 0} \quad \liminf\nolimits_{x \to 0}\\
    \limsup\nolimits_{x \to 0} \quad \projlim\nolimits_{x \to 0}\\
    \varliminf_{x \to 0} \quad \varlimsup_{x \to 0} \quad
    \varinjlim_{x \to 0} \quad \varprojlim_{x \to 0}
\end{align*}

\begin{table}[h]
    \begin{center}
        \begin{tabular}{llcc}\hline
            Name &Type &Typeset &Text\\\hline
            left parenthesis &\cc( &$($ &(\\
            right parenthesis &\cc) &$)$ &)\\
            left bracket &\cc[ or \cc{lbrack} &$[$ &[\\
            right bracket &\cc] or \cc{rbrack} &$]$ &]\\
            left brace &\cc\{ or \cc{lbrace} &$\{$ &\{\\
            right brace &\cc\} or \cc{rbrace} &$\}$ &\}\\
            backslash &\cc{textbackslash} & &\textbackslash\\
            backslash &\cc{backslash} &$\backslash$\\
            forward slash &\cc/ &$/$ &/\\
            left angle bracket &\cc{langle} &$\langle$\\
            right angle bracket &\cc{rangle} &$\rangle$\\
            vertical line &\cc| or \cc{vert} &$|$\\
            vertical line &\cc{textbar} & &\textbar\\
            vertical line &\cc{mid} &$\mid$\\
            double vertical line &\cc{|} or \cc{Vert} &$\Vert$\\
            left floor &\cc{lfloor} &$\lfloor$\\
            right floor &\cc{rfloor} &$\rfloor$\\
            left ceiling &\cc{lceil} &$\lceil$\\
            right ceiling &\cc{rceil} &$\rceil$\\
            upward &\cc{uparrow} &$\uparrow$\\
            double upward &\cc{Uparrow} &$\Uparrow$\\
            downward &\cc{downarrow} &$\downarrow$\\
            double downward &\cc{Downarrow} &$\Downarrow$\\
            up-and-down &\cc{updownarrow} &$\updownarrow$\\
            upper-left corner &\cc{ulcorner} &$\ulcorner$\\
            upper-right corner &\cc{urcorner} &$\urcorner$\\
            lower-left corner &\cc{llcorner} &$\llcorner$\\
            lower-right corner &\cc{lrcorner} &$\lrcorner$\\
            \hline
        \end{tabular}
        \caption{Standard delimiters.}
    \end{center}
\end{table}

: \begin{math}\int_{-\infty}^\infty \mathrm{e}^{-x^2}\,dx = \sqrt\pi\end{math}
\[
    \int_{-\infty}^\infty e^{-x^2}\,dx =\quad \sqrt\pi
\]

test $a \quad\pi$

\section{}
\begin{equation}
    \label{E:int}
    \int_{-\infty}^\infty e^{-x^2}\,dx = \sqrt\pi
\end{equation}

{\itshape see~\eqref{E:int} see~(\ref{E:int})}

Get: $a + b$, $a - b$, $-a$, $a / b$, $a b$, $ab$

And: $a \cdot b$, $a \times b$, $a \div b$, $a \mod b$
\[
    \frac{1 + 2x}{x + y + xy}
\]
And: $\dfrac{1 + 2x}{x + y + xy}$
\[
    \tfrac{1 + 2x}{x + y + xy}
\]
And: $\frac{1 + 2x}{x + y + xy}$
And: $\begin{bmatrix}\dfrac{1 + 2x}{x + y + xy}&\dfrac{3 + a^2}{4+b}\\\dfrac{1}{2}&\dfrac{3}{4}\end{bmatrix}$

See: $f'(x)$, $f^{\prime 2}(x)$, $f'^2(x)$, ${f'}^2(x)$

Neither ${}^\dagger$, here $\sb\dagger$, nor $\sp\dagger$ there

Testing $1 * 2 * \vdots$

\[
    \oint\limits_{-\infty}^\infty e^{-x^2}\,dx = \sqrt\pi
\]

\[
    \sqrt{1+\sqrt{1 + \frac{1}{2}\sqrt{1 + \frac{1}{3}\sqrt{1 + \frac{1}{4}\sqrt{1 + \frac{1}{5}\sqrt{1 + \cdots}}}}}}
\]

\noindent
In $\sqrt[g]5$\\
In $\sqrt[\leftroot2 \uproot2 g]5$\\
In $\sqrt[\uproot2 g]5$\\
In $\sqrt[5]5$\\
In $\sqrt5$
\[
    A = \{ x \mid x \in X_i, \text {for some} i \in I \}
\]
This
\[
    a_{\text{left}} + 2 = a_{\text{right}}
\]
\[
    a_{\textnormal{left}} + 2 = a_{\text{right}}
\]

\begin{thm}
    The constant $a_{\text{right}}$ is recursive in a.
\end{thm}
\begin{thm}
    The constant $a_{\text{\upshape right}}$ is recursive in a.
\end{thm}
\begin{thm}
    The constant $a_{\text{\bfseries right}}$ is recursive in a.
\end{thm}

Note $a_l$ vs $a_{\text l}$
\end{document}
