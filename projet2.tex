\documentclass[12pt]{article}

\usepackage{amsmath,amsthm}
\usepackage{amsfonts}
\usepackage[psamsfonts]{amssymb}
\usepackage{palatino,euler}
\usepackage{graphicx}
\usepackage{hyperref}
\usepackage[french]{babel}
\usepackage{gensymb}
\usepackage{siunitx}
\usepackage[margin=1.25in]{geometry}
\usepackage{tikz}

\newcommand\critical{\SI{3.98}\celsius}
\numberwithin{figure}{section}
\numberwithin{table}{section}

\title{MAT1460\\[3ex]Fermeture de la patinoire naturelle du Lac aux Castors}
\author{Chen, Siying\\[1ex] Labont\'e, Pierre-Luc\\[1ex]Higgins, Philip\\[1ex] Grenier, David}
\date{}
\begin{document}

\maketitle
\thispagestyle{empty}
\vfill
\begin{center}
Universit\'e de Montr\'eal

\today
\end{center}
\clearpage

\tableofcontents
\clearpage
\section{Introduction}

Au cours des derni\`eres ann\'ees, des travaux d'excavation ont \'ete effectu\'es par la Ville de
Montr\'eal pour r\'egler un probl\`eme de prolif\'eration d'algues. Toutefois, la patinoire
hivernale naturelle est d\'esormais ferm\'ee en vertu d'un accident hivernal reli\'e \`a la
qualit\'e de la glace. Les citoyens ne pourront d\'esormais plus jouir de cette activit\'e au Lac
aux Castors.

Le r\'echauffement plan\'etaire \'etant d\'esormais une r\'ealit\'e scientifiquement accept\'ee, les
gestionnaires ont bl\^am\'e la situation sur celui-ci. Une cons\'equence directe des
travaux est que la profondeur du lac a augment\'e de fa\c con significative. Il est donc possible
qu'il soit tout simplement plus difficile d'obtenir une \'epaisseur de glace suffisante en raison du
volume d'eau plus important \`a refroidir.

Nous avons donc construit certains mod\`eles permettant de v\'erifier l'impact des facteurs
mentionn\'es sur la qualit\'e de la glace.

\section{Impact de la temp\'erature}

Nous ne croyons pas \^etre en mesure de r\'epondre de fa\c con satisfaisante \`a la r\'ealit\'e du
r\'echauffement plan\'etaire. Nous pensons toutefois v\'erifier le bienfond\'e de l'argument de la
Ville de Montr\'eal sur la cause ultime de l'accident de l'hiver 2016-2017, survenu apr\`es les
travaux de r\'efections.

Nous allons d'abord d\'eterminer s'il y a une augmentation statistiquement significative dans les
temp\'eratures enregistr\'ees aux cours des ann\'ees r\'ecentes. Il est raisonnable de croire que la
condition de la glace d\'epend des temp\'eratures ext\'erieures et nous allons donc aussi tenter de
mesurer le niveau de corr\'elation entre les temp\'eratures observ\'ees et les donn\'ees d'ouverture
des patinoires de l'\^Ile de Montr\'eal~\cite{PatHist}.

\subsection{Hypoth\`eses}

Nos hypoth\`eses sont principalement ax\'ees sur la nature des donn\'ees qui nous sont disponibles:

\begin{enumerate}
    \item Nous supposons que les temp\'eratures ext\'erieures sur l'\^Ile de Montr\'eal sont
        uniform\'ement distribu\'ees;
    \item Lorsque les donn\'ees sont absentes en d\'ebut d\'ecembre et fin mars, nous supponsons que
        les patinoires sont ferm\'ees puisque hors-saison;
    \item Nous supposons les donn\'ees collectives sur les conditions des patinoires sont fiables;
    \item Nous supposons que l'\'ecart-type de temp\'erature donn\'ee par le minist\`ere est
        l'\'ecart-type r\'eel des temp\'eratures d'hivers avec lesquels nous comparons~\cite{AvgTemp};
    \item On suppose que s'il y a effectivement r\'echauffement plan\'etaire que ce dernier
        n'affecte pas significativement l'\'ecart-type.
\end{enumerate}

\subsection{Augmentation de la temp\'erature}

En utilisant les donn\'ees de temp\'eratures disponible~\cite{TempHist}, nous avons effectu\'e un
test de diff\'erence de moyennes pour d\'eterminer si les temp\'eratures moyennes des 10 derni\`eres
ann\'ees \'etaient semblable \`a celles des 30 derni\`eres ann\'ees~\cite{MeteoTemp}. La table
\ref{vars:hyptest} en donne les variables.

\begin{table}
    \centering
    \begin{tabular}{|l|l|r|}\hline
        Variable &Description &Valeur\\\hline
        $\mu_T$ &Moyenne de temp\'erature d\'ecembre-mars, 1981-2010 &-\SI{6.80}\celsius\\\hline
        $\mu_t$ &Moyenne \'echantillonnale d\'ecembre-mars &-\SI{5.06}\celsius\\\hline
        $\sigma_T^2$, $\sigma_t^2$ &\'Ecart-type donn\'e par le minist\`ere &\SI{2.25}\celsius\\\hline
        $n_T$ &Ann\'ees de temp\'eratures historique enregistr\'ees &30\\\hline
        $n_t$ &Ann\'ees de temp\'eratures r\'ecentes &10\\\hline
    \end{tabular}
    \caption{Test d'hypoth\`ese - \'echantillon 2008-2017}\label{vars:hyptest}
\end{table}

\begin{align*}
    H_0&: \mu_T - \mu_t = 0\\
    H_A&: \mu_T - \mu_t < 0\\
    p_\text{valeur} &= \frac{\mu_T - \mu_t}{\sqrt{\frac{\sigma_T^2}{n_T} + \frac{\sigma_t^2}{n_t}}} = \frac{\mu_T - \mu_t}{\sigma_T\sqrt{\frac1{n_T}+\frac1{n_t}}}\\
                    &= \frac{-6.8 + 5.06}{\sqrt{2.25}\sqrt{\frac1{30}+\frac1{10}}} = -3.17
\end{align*}

Notre statistique de test \'etant tr\`es faible, nous \'ecartons l'hypoth\`ese que les temp\'eratures
r\'ecentes sont dans les normes hivernales. Nous ne pr\'etendons rien ici sur le r\'echauffement
plan\'etaire, mais seulement que les temp\'eratures des 10 derni\`eres ann\'ees sont
significativement plus \'elev\'ees.

\subsection{Corr\'elation des jours d'ouverture des patinoires}

Nous avons aussi extrait les donn\'ees d'ouverture des patinoires de l'\^Ile de Montr\'eal. Nous
avons concentr\'e nos efforts sur les donn\'ees des saisons 2008 \`a 2013, ann\'ees pour lesquelles
nous avions les donn\'ees les plus robustes et ce, pour les mois de D\'ecembre \`a Mars.

Nous avons \'etabli deux mesures (voir table \ref{data:predict}) pour v\'erifier la corr\'elation
entre les temp\'eratures observ\'ees et l'ouverture des patinoire. La premi\`ere est le nombre de
jours o\`u la temp\'erature maximale est d'au plus \SI0\celsius. La seconde est plus complexe, on ne
comptabilisera une journ\'ee que si les crit\`eres suivants sont satisfaits:

\begin{enumerate}
    \item La temp\'erature moyenne de la journ\'ee est d'au plus \SI0\celsius;
    \item Une surface n'est patinable que si la journ\'ee est indirectement pr\'ec\'ed\'ee de deux
        nuits cons\'ecutives de temp\'eratures minimales d'au plus -\SI{10}\celsius;
    \item Indirectement signifie que la journ\'ee n'est pas s\'epar\'ee des nuits froides par
        trois jours cons\'ecutifs de temp\'erature moyenne sup\'erieure \`a \SI0\celsius.
\end{enumerate}

\begin{table}
    \centering
    \begin{tabular}{|l|r|r|r|r|}\hline
        Ann\'ee &Temp moyenne &Pr\'ediction 1 &Pr\'ediction 2 &Ouvertures\\
                &(\si\celsius) & & &(nombre)\\\hline
        2017 &-4.92 &59 &82 &\\\hline
        2016 &-3.96 &52 &70 &\\\hline
        2015 &-6.69 &74 &89 &\\\hline
        2014 &-6.59 &72 &90 &\\\hline
        2013 &-5.05 &58 &61 &\\\hline
        2012 &-2.57 &50 &65 &57\\\hline
        2011 &-4.78 &66 &69 &56\\\hline
        2010 &-5.26 &65 &69 &55\\\hline
        2009 &-5.53 &72 &77 &62\\\hline
        2008 &-5.26 &61 &91 &62\\\hline
        Moyenne &-5.06 &62.0 &76.3 &58.4\\\hline
    \end{tabular}
    \caption{Pr\'edictions 1 \& 2 et jours d'ouverture r\'eel}\label{data:predict}
\end{table}

On voit en figure \ref{predictions} la droite des moindres carr\'es entre nos deux pr\'edictions et
les temp\'eratures moyennes saisonni\`eres. En particulier, la pr\'ediction 1 \`a une corr\'elation
forte n\'egative $R = -0.892$.

\begin{figure}[h]
    \centering
    \includegraphics[scale=0.5]{Prediction1.png}
    \includegraphics[scale=0.5]{Prediction2.png}
    \caption{Pr\'edictions 1 \& 2 contre les temp\'eratures moyennes saisonni\`eres}\label{predictions}
\end{figure}

Nos pr\'edicteurs \'etant reli\'es aux temp\'eratures moyennes saisonni\`eres, nous pensons que
s'ils sont efficaces pour pr\'edire les jours d'ouverture r\'eels, nous aurons une bonne mesure
de l'impact des augmentations de temp\'eratures.

\begin{table}[h]
    \centering
    \begin{tabular}{|l|l|r|}\hline
        Variable &Description &Valeur\\\hline
        $n_1$, $n_2$ &Ann\'ees de donn\'ees de patinoires ouvertes &5\\\hline
        $s$ &Estimateur de la diff\'erence de moyenne &\\\hline
        $s_1$, $s_2$ &\'Ecart-type \'echantillonnal &\\\hline
        $\mu_1$ &Moyenne du pr\'edicteur 1 &\\\hline
        $\mu$ &Moyenne des jours d'ouverture &\\\hline
    \end{tabular}
    \caption{Variables pour les ann\'ees de donn\'ees pertinentes 2008-2012}\label{vars:predict}
\end{table}

Le r\'esultat d'un test d'hypoth\`ese d'une diff\'erence de moyennes est indiqu\'e \`a l'\'equation
\eqref{test:days}. Notre \'echantillon \'etant petit, nous avons effectu\'e un test de Student,
soumis aux variables de la table \ref{vars:predict}. On peut donc accepter que notre $1^\text{er}$
pr\'edicteur est raisonnable pour pr\'edire le nombre de jours r\'eel d'ouverture.

\begin{align}
    H_0 &: \mu_1 - \mu = 0\notag\\
    H_A &: \mu_1 - \mu \ne 0\notag\\
    s &= \sqrt{\frac{(n_1-1)s_1^2 + (n_2-1)s_2^2}{n_1+n_2-2}} = \sqrt{39}\notag\\
    T_{\mu_1-\mu} &= \frac{\mu_1 - \mu}{s\sqrt{\frac1{n_1}+\frac1{n_2}}} = 1.114\label{test:days}
\end{align}

La droite des moindres carr\'es de la pr\'ediction 1 est donn\'ee par l'\'equation \eqref{eq:y}.
Cette derni\'ere nous dit que l'impact d'une augmentation de temp\'erature moyenne annuelle d'un
\si{\celsius} se traduira par une perte de 6.72 jours de patinage par ann\'ee.

\begin{equation}
    y = -6.72x + 31.2\label{eq:y}
\end{equation}

Les mod\`eles actuels nous disent que la temp\'erature globale devrait augmenter de \SI4{\celsius}
d'ici 2100~\cite{GlobalRaise} relativement \`a la temp\'erature pr\'e-industrielle. Toutefois
l'impact Canadien d'une telle augmentation serait le double de l'impact global~\cite{CanadaRaise}.
Nous interpr\'etons cette derni\`ere comme une augmentation d'environ \SI6{\celsius} au cours des
prochains 80 ans et donc \si{0.75}{\celsius} par d\'ec\'enie.

\subsubsection{Validit\'e des r\'esultats}

Il n'est pas surprenant que notre premier pr\'edicteur soit fortement corr\'el\'e avec les moyennes
de temp\'erature annuelles. Quoique simple, ils donnent tous deux des valeurs raisonnables si bien
que nous n'avons pas cru n\'ecessaire de leurs appliquer un correctif. Nous nous fions toutefois sur
la communaut\'e scientifique en ce qui attrait au r\'echauffement plan\'etaire. Cette derni\`ere
question \'etant beaucoup plus sophistiqu\'ee, nous ne parlerons de l'impact sur la qualit\'e des
patinoires qu'en supposant leurs conclusions correctes.

\section{Impact de la profondeur}

Nous discuterons ici de l'effet de la profondeur du plan d'eau sur les conditions de glace. En
particulier, nous cherchons \`a savoir si l'excavation du lac~\cite{Lac} \`a contribu\'e de fa\c con
importante \`a la fermeture de la patinoire.

\subsection{Hypoth\`eses}

Afin de d\'eterminer l'impact de la profondeur sur la formation de la glace, nous allons encadrer la
question des simplifications suivantes:

\begin{enumerate}
    \item Nous allons consid\'erer l'air ext\'erieur comme un puit de chaleur vaste dont la
        temp\'erature ne sera pas affect\'ee par la chaleur du lac;
    \item Nous allons aussi traiter le sol comme une source de chaleur in\'epuisable;
    \item Nous n'allons \'evaluer que la perte d'\'energie du lac en terme de la conduction thermique
        et ignorer les contributions en radiation et \'evaporation~\cite{Evap};
    \item Nous supposons la temp\'erature ext\'erieure constante;
    \item Nous supposons aussi la temp\'erature du sol constante et uniform\'ement distribu\'ee;
    \item Nous ne connaissons pas les conductivit\'es thermiques en surface et au fond du lac, mais
        nous allons les consid\'erer respectivement identiques pour nos deux lacs;
    \item Nous ne connaissons pas la nature de l'interface en surface et au fond du lac pour le
        calcul de flux thermique et son \'epaisseur. Nous allons toutefois les consid\'erer
        identiques pour nos deux lacs.
\end{enumerate}

\subsubsection{Temp\'erature ext\'erieure constante}

Pour obtenir une r\'eponse r\'ealiste, une simulation ou une utilisation judicieuse de probabilit\'es
nous permettrait d'inclure une fluctuation de temp\'erature. En particulier, nous savons que la
chaleur transmise entre deux corps est proportionnelle \`a la diff\'erence de temp\'erature entre
ces derniers~\cite{Fourier}, il serait donc pr\'eferable de ne pas n\'egliger cette composante.

Toutefois, ce m\'ecanisme est identique pour la chaleur d\'egag\'ee en surface et celle re\c cue par
le fond du lac. S'il y a une diff\'erence importante de variation de temp\'erature, le lac plus
profond recevra d'avantage de chaleur du sol et il n'est pas clair que d'ajouter une temp\'erature
ext\'erieure variable va am\'eliorer le mod\`ele.

Puisque nous \'evaluerons la diminution de temp\'erature d'un lac par la chaleur d\'egag\'ee par le
plan d'eau \`a l'air environnant, il s'ensuit que la temp\'erature chutera en surface. La capacit\'e
thermique~\cite{CapTherm} massique du sol est inf\'erieure a celle de l'eau, toutefois le volume de
la terre est de loin sup\'erieur \`a celle d'un lac. Il semble que la temp\'erature du sol ne varie
que de \SI2{\celsius}~\cite{QuoraTemp} au cours d'une ann\'ee et de \SI{0.3}{\celsius} \`a tous les
dix m\^etre de profondeur. Nous choisissons \SI8{\celsius}~\cite{GeoTemp} comme temp\'erature du
sol.

\subsection{La convection thermique}

Il est connu que la densit\'e de l'eau augmente lorsque la temp\'erature~\cite{WaterDensity}
diminue, atteignant son maximum \`a \critical{} ce qui introduit un processus de convection
thermique~\cite{ConvNat}.

\begin{figure}[h]
    \centering
    \includegraphics[scale=0.7]{WaterDensity.png}
    \caption{Densit\'e de l'eau}\label{water-density}
\end{figure}

Nous allons donc consid\'erer qu'un lac dont la temp\'erature est sup\'erieure \`a \critical va
voir sa temp\'erature diminuer de fa\c con uniforme. D\'eterminer l'impact de la profondeur d'un
lac pour diminuer sa temp\'erature \`a \critical{} revient donc \`a mesurer le temps n\'ecessaire
pour obtenir les pertes de chaleurs respectives.

\begin{figure}
    \centering
    \begin{tikzpicture}[scale=0.7]
        \shade[top color=blue!70,bottom color=brown] (0,3) -- (0,2) arc (180:360:3 and 0.5) -- (6,3)
        arc (0:180:3 and 0.5);
        \draw [<->] (3,3) -- (5.9,3) node [midway, above, scale=0.7] {$r$};
        \draw [<->] (6.3,3) -- (6.3,2) node [midway, right, scale=0.7] {$p_1$};
        \draw (3,3) ellipse (3 and 0.5);
        \draw (0,3) -- (0,2);
        \draw (6,3) -- (6,2);
        \draw [dashed] (0,2) arc (180:0:3 and 0.5);
        \draw (0,2) arc (180:360:3 and 0.5);
        \draw[line width=0.5mm, ->] (1.8,2.7) arc (95:445:1cm and 0.45cm);
        \draw[line width=0.5mm, ->] (4.2,2.7) arc (85:-260:1cm and 0.45cm);

        \shade[top color=blue!70,bottom color=brown] (7,3) -- (7,-1) arc (180:360:3 and 0.5) -- (13,3)
        arc (0:180:3 and 0.5);
        \draw [<->] (10,3) -- (12.9,3) node [midway, above, scale=0.7] {$r$};
        \draw [<->] (13.3,3) -- (13.3,-1) node [midway, right, scale=0.7] {$p_2$};
        \draw (10,3) ellipse (3 and 0.5);
        \draw (7,3) -- (7,-1);
        \draw (13,3) -- (13,-1);
        \draw [dashed] (7,2) arc (180:360:3 and 0.5);
        \draw [dashed] (7,-1) arc (180:0:3 and 0.5);
        \draw (7,-1) arc (180:360:3 and 0.5);
        \draw[line width=0.5mm, ->] (8.8,2.7) arc (95:445:1cm and 2cm);
        \draw[line width=0.5mm, ->] (11.2,2.7) arc (85:-260:1cm and 2cm);
    \end{tikzpicture}
    \caption{Convection \`a plus de \critical}\label{water-convection}
\end{figure}

\subsection{La conduction thermique (ou diffusion)}

Comme pr\'ec\'edemment mentionn\'e, la temp\'erature du sol ne varie que de \SI{0.3}{\celsius} par
10 m\`etres de profondeur. Cette variation est lin\'eaire dans un solide \`a \'etat stable.

Remarquons aussi qu'en dessous de \critical, la densit\'e de l'eau se met \`a diminuer avec une
diminution de temp\'erature. Le processus de convection devient alors stable~\cite{HydroStab} et la
temp\'erature ne varie plus de fa\c con uniforme. On pense que la temp\'erature variera \`a
l'int\'erieur du corps par diffusion de chaleur, la profondeur aura donc quand m\^eme un impact.
Toutefois, il n'est plus n\'ecessaire \`a cette \'etape de refroidir le lac dans son ensemble pour
obtenir une formation de glace \`a la surface. Le refroidissement du lac doit donc \^etre trait\'e
diff\'eremment apr\`es l'atteinte de la temp\'erature uniforme de \critical.

\subsection{Au dessus de \SI4\celsius}

Nous allons d\'eterminer l'\'evolution, au dessus de \SI4{\celsius}, de la temp\'erature de deux
lacs id\'ealis\'es de profondeurs diff\'erentes. Nous appelerons ci-apr\`es les lacs 1 et 2 comme
\'etant le lac de r\'ef\'erence et le lac excav\'e, soumis aux variables suivantes:

\begin{table}[h]
    \centering
    \begin{tabular}{|l|l|l|}\hline
        Variable &Description &Unit\'e\\\hline
        $Q$ &Chaleur &\si\joule\ (\si{\kilogram.\square\meter\per{\square\second}})\\\hline
        $T$ &Temp\'erature &\si{\kelvin}\\\hline
        $c_p$ &Capacit\'e thermique massique &\si{\joule\per{\kelvin\,\kilogram}}\\\hline
        $m$ &Masse &\si\kg\\\hline
        $C$ &Capacit\'e thermique &\si{\joule\per\kelvin}\\\hline
        $K$ &Conductivit\'e thermique &\si{\joule\per{\meter\,\second\,\kelvin}}\\\hline
        $\rho$ &Densit\'e de la substance &\si{\kilogram\per{\square\meter}}\\\hline
        $V$ &Volume &\si{\cubic\meter}\\\hline
        $r$ &Rayon d'un lac cylindrique &\si\meter\\\hline
        $A$ &Superficie &\si{\square\meter}\\\hline
        $p$ &Profondeur d'un lac cylindrique &\si\meter\\\hline
        $t$ &Temps &\si\second\\\hline
    \end{tabular}
    \caption{Variables}
\end{table}

Lorsqu'un corps subit une variation de temp\'erature l'energie transf\'er\'ee
(chaleur)~\cite{Q-equation} nous est donn\'ee par l'\'equation \eqref{eq:heat}.

\begin{align}
    Q &= C\Delta T\label{eq:heat}\\
    &= c_pm\Delta T\notag\\
    &= c_p\rho V\Delta T\notag\\
    &= c_p\rho (p\pi r^2p)\Delta T &&\text{(pour un lac cylindrique)}\notag
\end{align}

La capacit\'e thermique d'un corps $(C)$ est une propri\'et\'e extensive~\cite{Extensive}, sa valeur
est directement proportionnelle \`a la masse du syst\`eme. Si la masse du second lac est le triple
du premier, il devra perdre trois fois plus d'\'energie pour atteindre la m\^eme temp\'erature.

La figure \ref{water-convection} nous montre un second lac expos\'e \`a la m\^eme surface
ext\'erieure. Sa superficie en contact avec le sol que est toutefois un peu plus importante. Nous
allons d'abord \'evaluer la chaleur \`a perdre pour atteindre \critical{} et ensuite le temps requis
pour perdre cette chaleur.

Les donn\'ees officielles sur la superficie et la profondeur du Lac aux Castors semblent difficile
\`a obtenir. Nous avons toutefois r\'eussi \`a estimer la superficie du lac \`a environ
\SI{18500}{\square\meter} (voir figure \ref{google-castor}). On nous informe aussi, qu'en vertu des
travaux, le lac est pass\'e d'une profondeur d'environ 2 \`a 7 m\`etres~\cite{Lac-Castor}.

\begin{figure}[h]
    \centering
    \includegraphics[scale=0.5]{Superficie.png}
    \caption{Superficie du Lac aux Castors}\label{google-castor}
\end{figure}

Nous simplifions le mod\`ele du lac par deux cylindres, respectant la superficie du Lac aux Castors,
aux profondeurs respectives de 2 et 7 m\`etres. La table \ref{LacParams} donne les param\`etres que
nous avons calcul\'es.

\begin{table}
    \centering
    \begin{tabular}{|l|l|r|}\hline
        Param\`etre &Description &Valeur\\\hline
        $c_p$ &Capacit\'e thermique massique de l'eau &\SI{4185.5}{\joule\per{\kelvin\,\kilogram}}\\\hline
        $T_\text{sol}$ &Temp\'erature du sol &\SI{281.15}\kelvin\ (\SI8\celsius)\\\hline
        $T_\text{air}$ &Temp\'erature de l'air &\SI{261.15}\kelvin\ (\SI{-12}\celsius)\\\hline
        $T_\text{init}$ &Temp\'erature initiale de l'eau &\SI{281.15}\kelvin\ (\SI8\celsius)\\\hline
        $T_\text{cible}$ &Temp\'erature \`a hydrostabilit\'e &\SI{277.13}\kelvin\\\hline
        $A_{lac}$ &Superficie du lac &\SI{18500}{\square\meter}\\\hline
        $p_1,p_2$ &Profondeur des lacs 1 et 2 &\SI2\meter, \SI7\meter\\\hline
        $\rho$ &Densit\'e de l'eau &\SI1{\kilogram\per{\cubic\meter}}\\\hline
        $K$ &Conductivit\'e thermique de surface &inconnue\\\hline
        $x$ &\'Epaisseur de l'interface de surface &inconnue\\\hline
        $\alpha$ &Ratio des constantes du fond contre surface &param\`etre\\\hline
    \end{tabular}
    \caption{Param\`etres pour la mod\'elisation \`a plus de \critical{}}\label{LacParams}
\end{table}

On trouve les pertes de chaleur n\'ecessaires de nos deux lacs:

\begin{align*}
    r &= \sqrt{\frac{18500}\pi} = \SI{76.738}\meter\\
    C_1 &= c_pm_1 = c_p\rho p_1\pi r^2 = \num{1.549e8}\si{\joule\per\kelvin}\\
    C_2 &= c_pm_2 = c_p\rho p_2\pi r^2 = \num{5.420e8}\si{\joule\per\kelvin}\\
    \Delta T &= T_\text{cible} - T_\text{init} = \SI{-4.02}\kelvin\\
    Q_1 &= C_1\Delta T = -\num{6.226e8}\si\joule\\
    Q_2 &= C_2\Delta T = -\num{2.179e9}\si\joule
\end{align*}

On doit maintenant \'evaluer la chaleur \'emise par la surface et re\c cue par le fond du lac.
L'\'equation du flux thermique \eqref{eq:flux} n'est normalement pas applicable
directement~\cite{HeatFlow} puisqu'elle est typiquement utilis\'ee entre deux syst\`emes de masse et
capacit\'e thermique massique identiques. Aussi, la structure des interfaces entre la surface de
l'eau et l'air ambiant, ainsi que celle du fond du lac, nous sont inconnues. Nous ne pensons pas
pouvoir trouver la conductivit\'e thermique appropri\'ee ainsi que le gradient de temp\'erature
r\'eel.

\begin{align}
    \frac{\Delta Q}{\Delta t} = -KA\frac{\Delta T}x \label{eq:flux}
\end{align}

Nous allons donc absorber les constantes dans la variable d'int\'egration. Nous n'obtiendrons donc
pas une unit\'e de temps r\'eelle, mais nous allons toutefois \^etre en mesure d'\'evaluer la
diff\'erence relative entre nos deux syst\`emes.

\begin{align}
    A_1 &= 2\pi r p_1 + \pi r^2 = \num{1.946e4}\si{\square\meter}\notag\\
    \frac{dQ_1}{dt} &=
        \underbrace{-KA_\text{lac}\frac{T_\text{air}-T_1(t)}x\
            -\alpha KA_1\frac{T_\text{fond}-T_1(t)}x}_{\tau = \frac Kxt}\notag\\
    \frac{dQ_1}{d\tau} &= -A_\text{lac}T_\text{air}-\alpha A_1 T_\text{fond} + T_1(\tau)(A_\text{lac}+ \alpha A_1)\label{eq:diff}\\
    \text{o\`u } T_1(\tau) &= T_\text{init} - \frac{Q_1(\tau)}{C_1} \label{eq:temp}&&\text{(par l'\'equation \eqref{eq:heat})}
\end{align}

En substituant \eqref{eq:temp} dans l'\'equation \eqref{eq:diff} on obtient une \'equation
diff\'erentielle \`a variables s\'eparables. En donnant les conditions initiales, ($Q(0) = 0$) on
trouve $Q_1(\tau)$ et $Q_2(\tau)$. Apr\`es une substitution dans \eqref{eq:temp}, on obtient les
\'equations nous donnant l'\'evolution de la temp\'erature en fonction de notre temps $\tau$.

\begin{align*}
    T_1(\tau) &=
        T_\text{init} +
            \left(1-e^{\textstyle -t\frac{\alpha A_1+A_\text{lac}}{C_1}}\right)
            \frac
                {(\alpha A_1+A_\text{lac})T_\text{init} - A_\text{lac} T_\text{air} - \alpha A_1 T_\text{fond}}
                {\alpha A_1+A_\text{lac}}\\
    T_2(\tau) &=
        T_\text{init} +
            \left(1-e^{\textstyle -t\frac{\alpha A_2+A_\text{lac}}{C_2}}\right)
            \frac
                {(\alpha A_2+A_\text{lac})T_\text{init} - A_\text{lac} T_\text{air} - \alpha A_2 T_\text{fond}}
                {\alpha A_2+A_\text{lac}}\\
\end{align*}

Ces derni\`eres \'etant inversibles, on a pu trouver le temps $\tau$ pour atteindre la perte de
chaleur recherch\'ee pour diverses valeurs de $\alpha$. On trace ainsi les courbes des chutes de
temp\'eratures pour les deux lacs.

\begin{figure}[!h]
    \centering
    \includegraphics[scale=0.5]{Alpha.png}
    \caption{Chutes de temp\'eratures pour $\alpha \in \{0.5, 3, 4\}$.}
\end{figure}
\clearpage

\subsubsection{Validit\'e des r\'esultats}

Remarquons d'abord que le comportement observ\'e est celui attendu, que le lac de plus grand volume
met un plus grand temps \`a refroidir, mais aussi, que plus la temp\'erature est faible, plus
l'apport en \'energie du sol est importante. En essayant des lacs de plus grande profondeur, on a
observe qu'il est possible que le lac n'atteigne jamais \critical{}. Les Grands Lacs, par exemple,
ne g\^elent jamais compl\`etment. Aussi, le mod\`ele r\'epond aussi bien \`a un changement des
conditions initiales de temp\'erature qu'\`a une variation de la constante $\alpha$. Cette
derni\`ere montre aussi l'influence d'un apport plus important de la chaleur du sol quand la
temp\'erature de l'eau diminue.

\subsection{De \SI4{\celsius} \`a \SI0\celsius}

Lorsque \critical{} est atteint globalement dans le lac, la densit\`e de l'eau augmente \`a mesure
que sa temp\'rature diminue. Cependant, cette variation de densit\'e est moins marqu\'e que la
lorsque l'eau passe de \SI{10}{\celsius} \`a \critical{}. Toutefois, ceci fait en sorte que le lac ne
peut pas se r\'echauffer globalement. La surface perd sa chaleur plus rapidement, car elle est
directement en contact avec l'air. Nous mod\'eliserons un processus de diffusion soumis aux
variables et param\`etres de la table \ref{t:4-0}.

\begin{table}[h]
    \centering
    \begin{tabular}{|l|l|l|r|}\hline
        Nom &Type &Description &Valeur\\\hline
        $T_i(t)$ &Var &Temp\'erature de la strate $i$ \`a l'it\'eration $t$ &\\\hline
        $\Delta T_i(t)$ &Var &Diff\'erence de temp\'erature avec la strate sup\'erieure &\\\hline
        $T_i(0)$ &Param &Temp\'erature initiale de la strate $i$ &\SI4{\celsius}\\\hline
        $w$ &Param &Taux d'\'echappement de chaleur &0.1\\\hline
        $N$ &Param &Nombre de strates &\\\hline
        $k$ &Param &Constante de proportionalit\'e &\\\hline
    \end{tabular}
    \caption{Param\`etres pour la mod\'elisation de \critical{} \`a \SI0{\celsius}}\label{t:4-0}
\end{table}

Nous mod\'elisons cette partie tel qu'illustr\'e dans la figure \ref{Strates}. Des strates d'eau
superpos\'ees vont repr\'esenter le comportement du lac. \`A l'instant 0, le lac est uniformément à
\SI4{\celsius}. Lors de la simulation, la strate la plus basse restera \`a temp\'erature
constante, repr\'esentant la couche d'eau inf\'erieure, en contact avec le sol.

\begin{figure}
    \centering
    \includegraphics[scale=0.5]{Strates.png}
    \caption{Transmission de la chaleur dans le mod\`ele it\'eratif}\label{Strates}
\end{figure}

Ensuite, \`a chaque it\'eration, chaque strate d\'egage de la chaleur \`a un rythme proportionnel
\`a l'\'ecart de temp\'erature entre ses surfaces connexes. La strate sup\'erieure est expos\'ee \`a
l'air \`a -\SI{12}{\celsius}. L'\'echange de chaleur entre l'interface air-eau, pour une
diff\'erence de temp\'erature donn\'ee, est proportionnelle \`a une constante $k$ des interfaces
eau-eau. Les temp\'eratures des strates \'evoluent donc selon les \'equations \eqref{eq:air} et
\eqref{eq:water} construite \`a partir de l'\'equation du flux de chaleur \eqref{eq:flux}.

\begin{align}
    \Delta T_i(t) &= T_i(t)-T_{i-1}(t)\notag\\
    T_N(t) &= 4\notag\\
    T_1(t) &= T_1(t-1) - k w(12 + T_1(t-1)) + w\Delta T_2(t-1)\label{eq:air}\\
    T_i(t) &= T_i(t-1) - w\Delta T_i(t-1) + w\Delta T_{i+1}(t-1) &&(1 < i < N)\label{eq:water}
\end{align}

Nous avons expos\'e le mod\`ele \`a une variation de profondeur via le param\`etre $N \in \{10, 14,
22, 26, 30, 34\}$ et la constante $k \in \{ 0.05, 0.075, 0.1, 0.15, 0.2 \}$. La
figure \ref{StratTemp} nous donne le nombre d'it\'erations n\'ecessaire pour atteindre
\SI0{\celsius} \`a la strate sup\'erieure.

\begin{figure}
    \centering
    \includegraphics[scale=0.7]{StratesTemp.png}
    \caption{Temps \'ecoul\'e de \critical{} \`a \SI0{\celsius} pour diff\'erentes profondeurs}\label{StratTemp}
\end{figure}

\subsubsection{Validit\'e des r\'esultats}

On peut observer que seulement lorsque la profondeur et le d\'egagement en surface sont tr\`es
faible la profondeur aura un impact. Nous expliquons ceci par le fait que si le lac est trop mince,
il ne repr\'esente pas le profil d'un lac normal, on a plus une situation o\`u l'eau repr\'esente
une couche isolante sur le sol. Par exemple, si nous avions seulement 3 strates ceci
repr\'esenterait un lac s\'epar\'e en deux temp\'eratures, ce qui manque de pr\'ecision pour
d\'ecrire les variations de temp\'erature dans le lac.

\section{Analyse}

L'analyse du probl\`eme fut compartiment\'e en deux sous-questions: l'\'etude de l'impact de la
temp\'erature ambiante l'ouverture des patinoires ext\'erieures et l'\'etude de l'impact de la
profondeur d'un lac sur la formation de glace en surface.

Les mod\`eles de la premi\`ere sous-question nous indiquent que l'augmentation de temp\'erature a
un impact raisonnable sur la condition des patinoires ext\'erieures et qu'on ne peut pas \'ecarter
d'embl\'e l'affirmation des gestionnaires de la ville.

Toutefois, la corr\'elation entre une augmentation d'un \si\celsius et le nombre de jours
d'ouverture n'est pas dramatique. En effet, il faudrait environ une d\'ecennie de r\'echauffement
plan\'etaire pour observer une perte de 6 jours d'ouverture de patinage ext\'erieur.

Les mod\`eles de la seconde sous-question nous indiquent que l'augmentation de la profondeur n'a pas
un impact \`a toutes les \'etapes du processus de refroidissement du lac. Au dessus de \critical, on
observe une diff\'erence majeure du temps n\'ecessaire pour atteindre la temp\'erature
hydrostatique. Au dessous de cette temp\'erature, la profondeur ne semble pas avoir une influence
significative sur le temps pris pour refroidir la surface au point de cong\'elation.

En effet, nous avons trouv\'e que d'avoir tripl\'e le volume d'eau du second lac augmente
proportionnellement l'\'energie n\'ecessaire pour refroidir le lac. \`A surface \'egale, il sera
in\'evitable que le temps n'\'ecessaire pour d\'egager plus de chaleur est sup\'erieur.

Nous pensions initialement que la profondeur aurait un impact sur la variation de temp\'erature en
dessous de \critical{}, toutefois notre mod\`ele indique clairement que la profondeur n'a
pratiquement aucun impact. Nous croyons que le comportement du processus de diffusion tend vers une
distribution lin\'eaire lors d'un \'etat \'eventuel stable, mais que la temp\'erature en surface
atteindra \SI0{\celsius} bien avant l'uniformisation de la temp\'erature. Ainsi, il est raisonnable
de croire que les strates sup\'erieurs jouent un r\^ole plus important que celles du fond, qui ont
un apport n\'egligeable.

Nous pensons qu'une fois le processus de formation de glace commenc\'e le second mod\`ele va
continuer de bien repr\'esenter l'\'evolution de temp\'erature du lac, ainsi la profondeur n'aura
toujours pas d'impact. En fait, nous avons trouv\'e une \'equation~\cite{eq:ice} caract\'erisant la formation de
la glace et cette derni\`ere ne d\'epend seulement des \'echanges de chaleur en surface, ce qui
appuie notre hypoth\`ese.

\subsection{Forces et faiblesses}

La plus grande force de notre mod\`ele est d'avoir s\'epar\'e le processus de refroidissement en
deux \'etapes distinctes. En particulier, s\'epar\'e cette t\^ache permet d'arriver \`a une
conclusion oppos\'ee \`a celle obtenue en ignorant le processus de convection.

Plusieurs de nos \'etapes sont param\'etrisable et il est possible d'exploiter le mod\`ele \`a des
environnements vari\'es, comme ceux qui ne sont pas sujets \`a la formation de glace. Il est aussi
possible de voir vers quelle temp\'erature un lac va converger avec de l\'eg\`eres modifications au
mod\`ele.

Par contre, lors de l'\'etape de convection, on a simplifi\'e le mod\`ele en lui attribuant une temp\'erature
uniforme. Ce saut est important puisqu'en r\'ealit\'e ce processus est beaucoup plus complexe et
pourrait \^etre mieux mod\'elis\`e par des courants.

Dans le mod\`ele de diffusion, on suppose que la strate inf\'erieur demeure \`a temp\'erature
constante de \SI4{\celsius} alors qu'elle subi plut\^ot un influt de chaleur du fond. Ceci fait en
sorte que la chaleur se d\'egage globalement de notre syst\`eme sans influx de chaleur ce qui nous
semble mal repr\'esenter la r\'ealit\'e.

Aussi, les pr\'edicteurs qualitatifs choisis dans le premi\`er mod\`ele sont arbitraires. Il n'est
pas certain que ces derniers repr\'esentent ad\'equatement la r\'ealit\'e compte tenu de
l'\'echantillon limit\'e sur les jours d'ouverture. Par ailleurs, il semble que la qualit\'e de ces
donn\'es \'etaient plut\^ot faible.

\subsection{Am\'eliorations possible}

Pour am\'eliorer les r\'esultats produits par le mod\`ele, nous pourrions trouver des donn\'es sur
les lacs du Qu\'ebec et trouver des valeurs optimales pour la constante de proportion et celles de
flux de chaleur. Cela nous permettrait d'avoir une unit\'e de temps r\'eele et pouvoir faire des
pr\'edictions.

Pour am\'eliorer le premier mod\`ele, nous pourrions aller chercher des donn\'ees de
diff\'erents provinces ou pays aux conditions climatiques semblables. Nous aurions une plus grande
quantit\'e de donn\'ees, potentiellement de meilleure qualit\'e, ce qui nous donnerait acc\`es \`a
une meilleure analyse statistique.

\section{Conclusion}

Le but de ce travail est de d\'eterminer la cause ultime de la fermeture de la patinoire du Lac aux
Castors. Nous avons vu l'impact qu'une augmentation de temp\'erature peut avoir sur les jours
d'ouverture des patinoires. Nous sommes toutefois arriv\'es \`a la conclusion que l'impact de la
profondeur du lac, dans un pr\'esent imm\'ediat, est dominante. Le volume a augment\'e de fa\c con
dramatique et la fermeture ne pourra\^it \^etre expliqu\'ee autrement. En effet, le r\'echauffement
climatique n'a un impact qu\`a longue \'echelle sur le nombre de jours d'ouverture des patinoires.
Il est plausible d'affimer que le nombre de jours annuel d'ouverture \`a Montr\'eal n'a pas assez
diminu\'e pour expliquer l'accident de la saison 2016-2017 menant \`a la fermeture de la patinoire.
Ainsi, ce sont les travaux d'excavation qui ont eu le plus grand impact sur la qualit\'e de la
glace.

Il pourra\^t \^etre int\'eressant d'explorer la profondeur id\'eale \`a excaver pour r\'egler le
probl\`eme d'algues tout en conservant la qualit\'e de glace n\'ecessaire pour prot\'eger la saison
de patinage.

\clearpage
\section{Bibliographie}
\begin{thebibliography}{Xyz}
    \bibitem{PatHist} \href{http://donnees.ville.montreal.qc.ca/dataset/patinoires-historique}
        {Patinoire - historique des conditions}
    \bibitem{AvgTemp} \href{http://www.mddelcc.gouv.qc.ca/climat/surveillance/classification.htm}
        {Classification climatologique des temp\'eratures}
    \bibitem{TempHist} \href{https://www.meteomedia.com/ca/api/sitewrapper/index?b=%2Fstatistics%2F&p=%2Fprevisions%2Fstatistiques%2Findex&url=%2Fstatistics%2Fcaqc0363%2Fmontreal%2F%2F%2F%3F}
        {M\'et\'eoM\'edia - Montr\'eal}
    \bibitem{MeteoTemp} \href{https://www.meteomedia.com/ca/api/sitewrapper/index?b=%2Fstatistics%2F&p=%2Fprevisions%2Fstatistiques%2Findex&url=%2Fstatistics%2Fcaqc0363%2Fmontreal%2F%2F%2F%3F}
        {Statistiques: Montr\'eal, Que\'ebec - M\'et\'eoM\'edia}
    \bibitem{GlobalRaise} \href{https://www.independent.co.uk/environment/global-warming-temperature-rise-climate-change-end-century-science-a8095591.html}
        {Worst-case global warming predictions}
    \bibitem{CanadaRaise} \href{https://www.canada.ca/fr/environnement-changement-climatique/services/changements-climatiques/science.html}
        {Science des changements climatiques}
    \bibitem{Lac} \href{https://www.ledevoir.com/politique/montreal/517828/patinoire-du-lac-aux-castors}
        {Montr\'eal abandonne la patinoire naturelle du lac aux Castors}
    \bibitem{Evap} \href{https://www.quora.com/How-does-evaporation-take-place-at-all-temperatures-whereas-boiling-takes-place-at-a-fixed-temperature-under-a-given-pressure}
        {How does evaporation take place at all temperatures}
    \bibitem{Fourier} \href{https://fr.wikipedia.org/wiki/Conduction_thermique#Loi_de_Fourier}
        {Loi de Fourier}
    \bibitem{CapTherm} \href{https://fr.wikipedia.org/wiki/Capacit%C3%A9_thermique_massique}
        {Capacit\'e thermique massique}
    \bibitem{QuoraTemp} \href{https://fr.quora.com/Quelle-est-la-temp%C3%A9rature-de-la-terre-%C3%A0-10m-et-et-20m}
        {Quelle est la temp\'erature de la terre}
    \bibitem{GeoTemp} \href{http://www.sblais.com/default.asp?idpage=2377&idpageparent=2315}
        {La g\'eothermie}
    \bibitem{WaterDensity} \href{http://www.open.edu/openlearn/science-maths-technology/the-oceans/content-section-3.2}
        {The density of fresh water and seawater}
    \bibitem{ConvNat} \href{https://fr.wikipedia.org/wiki/Convection_thermique#Convection_naturelle}
        {Convection naturelle}
    \bibitem{HydroStab} \href{https://fr.wikipedia.org/wiki/Gradient_thermique_adiabatique#Atmosph.C3.A8re_stable}
        {Stabilit\'e hydrostatique}
    \bibitem{Q-equation} \href{https://simple.wikipedia.org/wiki/Specific_heat#Usage}
        {Calculating heat}
    \bibitem{Extensive} \href{https://fr.wikipedia.org/wiki/Extensivit%C3%A9_et_intensivit%C3%A9_(physique)}
        {Extensivit\'e}
    \bibitem{Lac-Castor} \href{https://fr.wikipedia.org/wiki/Lac_aux_Castors_(Montr%C3%A9al)}
        {Lac aux Castors}
    \bibitem{HeatFlow} \href{https://en.wikipedia.org/wiki/Rate_of_heat_flow}
        {Rate of heat flow}
    \bibitem{eq:ice} \href{http://static1.1.sqspcdn.com/static/f/572109/18056158/1336339358653/thin+ice+growth.pdf?token=GFUmdBjFmXo1TBl8yAgChTAGQGA%3D}
        {Thin ice growth}
    \bibitem{Conductivity} \href{https://www.engineeringtoolbox.com/thermal-conductivity-d_429.html}
        {Thermal Conductivity of common Materials and Gases}
    \bibitem{AvgValue} \href{https://fr.wikipedia.org/wiki/Th%C3%A9or%C3%A8me_de_la_moyenne}
        {Th\'eor\`eme de la moyenne}
    \bibitem{NewtonLaw} \href{https://en.wikipedia.org/wiki/Newton%27s_law_of_cooling}
        {Newton Law of cooling}
    \bibitem{TempLinear} \href{https://fr.wikipedia.org/wiki/Conduction_thermique#Surfaces_planes_en_série}
        {Conduction Thermique - Profil des temp\'eratures}
    \bibitem{SpecHeat} \href{https://www.engineeringtoolbox.com/specific-heat-capacity-d_391.html}
        {Specific heat of common Substances}
\end{thebibliography}

\end{document}
