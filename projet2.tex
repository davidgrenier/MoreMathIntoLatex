\documentclass[12pt]{article}

\usepackage{amsmath,amsthm}
\usepackage{amsfonts}
\usepackage[psamsfonts]{amssymb}
\usepackage{palatino,euler}
\usepackage{graphicx}
\usepackage{hyperref}
\usepackage[french]{babel}
\usepackage{gensymb}
\usepackage{siunitx}

\begin{document}
\section{Impact de la profondeur}

En relation avec les travaux effectu\'es au Lac aux Castors, nous discuterons ici de l'effet de la
profondeur du plan d'eau sur les conditions de glace. En particulier, nous cherchons \`a savoir s'il est
possible que d'avoir creus\'e\cite{Lac} le lac \`a contribu\'e \`a la fermeture de la patinoire pour
assurer la s\'ecurit\'e du personnel et des citoyens.

\subsection{Hypoth\`eses}

Afin de d\'eterminer l'impact de la profondeur d'un plan sur la formation de la glace nous allons
encadrer la question des simplifications suivantes:

\begin{enumerate}
    \item Quoique la temp\'erature ext\'erieure peut varier, nous allons consid\'rer l'air comme un puit
        de chaleur infinie. C'est \`a dire que l'apport en chaleur qu'un seul lac contribue \`a la temp\'erature
        environnante n\'egligeable;
    \item Identiquement, l'apport du lac et de l'air \`a la temp\'erature du sol le sera aussi et nous
        allons consi\d'erer cette derni\`ere comme une source de chaleur infinie;
    \item Nous n'allons \'evaluer que la perte d'\'energie du lac en terme de la conduction thermique et
        supposer que les contributions en radiation et \'evaporation des deux lacs est identique;
    \item Nous allons supposer la temp\'erature ext\'erieure constante. Nous expliquons ce choix
        ci-dessous.
\end{enumerate}

Nous tenterons donc de d\'eterminer, pour un plan d'eau d'une surface de superficie fixe, de quelle fa\c
con une augmentation de la profondeur (et donc du volume d'eau) va prolonger la p\'eriode n\'ecessaire
pour refroidir le lac. On comparera donc deux corps d'eau id\'ealis\'es et d\'elimit\'es par la m\^eme
surface circulaire, l'un plus profond que l'autre.

\subsubsection{Temp\'erature ext\'erieure constante}

Pour d\'eterminer la r\'eponse de fa\c con r\'ealiste nous aurions besoin d'une simulation ou d'une
utilisation judicieuse de probabilit\'es pour inclure une variation \`a la temp\'erature. En particulier,
nous savons que la chaleur transmise est proportionnelle \`a la diff\'erence de temp\'erature entre les
deux corps\cite{Fourier} et il serait pr\'eferable de ne pas n\'egliger cette composante.

Notre intuition nous dit qu'un lac dont le volume d'eau est sup\'erieur demandera plus de temps pour
changer de temp\'erature et que la diff\'erence de temp\'erature sera plus grande lorsque la
temp\'erature ext\'erieure est basse. Toutefois il est important d'appr\'ecier que si tel est le cas,
lorsque la temp\'erature ext\'erieure passe au dessus de celle de nos deux lacs, la situation est
renvers\'e et le lac dont le volume est inf\'erieur verra sa temp\'erature remonter plus rapidement.

Nous jugeons donc qu'il est raisonnable de supposer la temp\'erature ext\'erieure constante. Les
composantes de gains et pertes en chaleur d\^u \`a la variation de temp\'erature vont possiblement
s'annuler. Si ce n'est pas le cas et que le lac plus profond a une diff\'erence de temp\'erature avec
l'ext\'erieure plus grande on aura tout de m\^eme une r\'eponse positive \`a la question (que le lac plus
profond prends plus de temps \`a geler) en ayant possiblement sur\'estim\'e l'impact de la profondeur.

\subsubsection{La convection thermique}

Puisque nous \'evaluerons la diminution de la temp\'erature d'un lac par la chaleur fournie par le plan
d'eau \`a l'air environnant il s'ensuit que la temp\'erature chutera en surface. Comme nous le verrons
plus loin, la capacit\'e thermique\cite{CapTherm} massique du sol est inf\'erieure a celle de l'eau,
toutefois le volume de la terre est de loin sup\'erieur \`a celle d'un lac et dans un contexte o\`u la
surface sera sujette au gel on pourra supposer que le sol sera une source de chaleur.

Une particularit\'e de l'eau est que la densit\'e de la phase solide est inf\'erieure \`a celle de la
phase liquide. De plus, la densit\'e de l'eau est maximale \`a $\SI{3.98}{\celsius}$ ce qui introduit
in\'evitablement un processus de convection thermique\cite{ConvNat}. Tant que la temp\'erature du lac
n'atteint pas $\SI{3.98}{\celsius}$ l'eau qui refroduit \`a la surface va chuter et sera remplac\'ee par
de l'eau plus chaude provenant pr\`es de la terre l\`a o\`u la temp\'erature est plus \'elev\'ee.

\clearpage
\begin{thebibliography}{Xyz}
    \bibitem{Lac}
        \href{https://www.ledevoir.com/politique/montreal/517828/patinoire-du-lac-aux-castors}
            {Montr\'eal abandonne la patinoire naturelle du lac aux Castors}
    \bibitem{Fourier} \href{https://fr.wikipedia.org/wiki/Conduction_thermique#Loi_de_Fourier}
        {Loi de Fourier}
    \bibitem{ConvNat} \href{https://fr.wikipedia.org/wiki/Convection_thermique#Convection_naturelle}
        {Convection naturelle}
    \bibitem{CapTherm} \href{https://fr.wikipedia.org/wiki/Capacit%C3%A9_thermique_massique}
        {Capacit\'e thermique massique}
    \bibitem{SpecHeat} \href{https://www.engineeringtoolbox.com/specific-heat-capacity-d_391.html}
        {Specific heat of common Substances}
\end{thebibliography}

\end{document}
