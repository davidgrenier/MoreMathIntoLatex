\documentclass{amsart}
\begin{document}
\noindent
This space has the following properties:
\begin{enumerate}
    \item Grade 2 Cantor\label{Cantor};
    \item Half-smooth Hausdorff\label{Hausdorff};
    \item[a] Missing\label{david}.
    \item Metrizably smooth\label{smooth}.
\end{enumerate}
Therefore, we can apply the main Theorem for \ref{Cantor}, \ref{Hausdorff} and \ref{smooth}.

\medskip
\noindent
We set out to accomplish a variety of goals:
\begin{itemize}
    \item To introduce the concept of smooth functions.
    \item To show their usefulness in differentiation.
    \item To point out the efficacy of using smooth functions in Calculus. The quick
        brown fox jumps over the lazy dog.
\end{itemize}

\medskip
\noindent
In this introduction, we describe the basic techniques:
\begin{description}
    \item[Chopped lattice] a reduced form of lattice;
    \item[Boolean triples] a powerful lattice construction;
    \item[Cubic extension] a subdirect power flattening the congruences.
\end{description}

\medskip
\begin{enumerate}
    \item First item of Level 1.
        \begin{enumerate}
            \item First item of Level 2.
                \begin{enumerate}
                    \item First item of Level 3.
                        \begin{enumerate}
                            \item First item of Level 4.
                            \item Second item of Level 4.\label{secondl4}
                        \end{enumerate}
                    \item Second item of Level 3.
                \end{enumerate}
            \item Second item of Level 2.
        \end{enumerate}
    \item Second item of level 1.
\end{enumerate}
Referencing the second item of Level 4: \ref{secondl4}

\medskip
\begin{enumerate}
    \item First item of Level 1.
        \begin{itemize}
            \item First item of Level 2.
                \begin{enumerate}
                    \item First item of Level 3.
                        \begin{itemize}
                            \item First item of Level 4.
                            \item Second item of Level 4.\label{enums}
                        \end{itemize}
                    \item Second item of Level 3.
                \end{enumerate}
            \item Second item of Level 2.
        \end{itemize}
    \item Second item of level 1.
\end{enumerate}
Referencing the second item of Level 4: \ref{enums}

\begin{flushright}
    The {\bfseries simplest}
    text environments set the
    printing style and size.\\
    The commands and the environments have similar names.
\end{flushright}
\end{document}
