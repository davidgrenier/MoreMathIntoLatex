\documentclass{amsart}
\begin{document}
See $\sqrt a + \sqrt b$ vs $\sqrt{\mathstrut a} + \sqrt{\mathstrut b}$ vs $\sqrt{\vphantom ba} + \sqrt b$

\[
    \Theta_i = \bigcup \bigl(\Theta(\overline{a \wedge b},
                                \overline{\vphantom ba} \wedge \overline b) \mid
                            a,\ b \in B_i \bigr)
                    \vee \bigcup \bigr( \Theta(\overline{a \vee b},
                                \overline{\vphantom ba} \vee \overline b) \mid
                            a,\ b \in B_i \bigr)
\]

    It is \emph{very important} that you memorize the integral
    $\smash{\frac1{\int f(x)\,dx}} = 2g(x) + C$,
    which will appear on the next test.

    It is \emph{very important} that you memorize the integral
    $\frac1{\int f(x)\,dx} = 2g(x) + C$,
    which will appear on the next test.

\[
    \int_{-\infty}^\infty e^{-x^2}\,dx = \sqrt\pi\tag{Int}
\]
\[
    \int_{-\infty}^\infty e^{-x^2}\,dx = \sqrt\pi\tag*{A--B}
\]

Variant:
\begin{equation}
    A^{[2]} \diamond B^{[2]} \cong
    (A \diamond B)^{[2]}\label{E:cong}
\end{equation}
\begin{equation}
    A^{\langle 2 \rangle} \diamond B^{\langle 2 \rangle}
    \equiv (A \diamond B)^{\langle 2 \rangle}\tag{\ref{E:cong}$'$}
\end{equation}
Subequations:
\begin{subequations}\label{E:joint}
    \begin{equation}
        A^{[2]} \diamond B^{[2]} \cong
        (A \diamond B)^{[2]}\label{E:cong}
    \end{equation}
    \begin{equation}
        A^{\langle 2 \rangle} \diamond B^{\langle 2 \rangle}
        \equiv (A \diamond B)^{\langle 2 \rangle}\label{E:equiv}
    \end{equation}
\end{subequations}
These equations~\eqref{E:joint} are~\eqref{E:cong} and~\eqref{E:equiv}
\end{document}
