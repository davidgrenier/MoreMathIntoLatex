\documentclass[12pt]{article}

\usepackage{amsmath,amsthm}
\usepackage{amsfonts}
\usepackage[psamsfonts]{amssymb}
\usepackage{palatino,euler} 
\usepackage[applemac]{inputenc}
\usepackage[french]{babel}
\usepackage[T1]{fontenc}
\usepackage{graphicx}
\usepackage{tikz}
\usepackage{hyperref}


\setlength{\topmargin}{0pt}
\setlength{\headheight}{0pt}
\setlength{\headsep}{0pt}
\setlength{\textheight}{612pt}
\setlength{\textwidth}{400pt}
\setlength{\marginparwidth}{0pt}
\setlength{\oddsidemargin}{36pt}
\setlength{\parindent}{0pt}


\begin{document}

\noindent{\bf \large MAT 1460 : rapport - devoir 1 - Cerner le probl\`eme}


\medskip
\noindent{\bf Pr\'enom Nom} : David Grenier

\noindent{\bf Date} : \today

\bigskip


\bigskip

\hrule height0.3pt

\bigskip


\noindent{\bfseries 1.} Mes interpr\'etations  du mot \og optimal \fg{} sont:
\begin{enumerate}
    \item La ligne qui se rentabilise le plus rapidement:
    \item[] Il s'ag\^it d'une certaine mesure de succ\'es et pourra\^it favoriser l'ouverture d'une ligne subs\'equente;
    \item La ligne qui \'eliminerait le plus d'automobiliste sur les routes:
    \item[] Sous cette interpr\'etation, une ligne ciblant les automobilistes traversant les ponts pourra\^it \^etre pr\'ef\'erable;
    \item La ligne qui convertirait le plus d'utilisateurs d'autobus:
    \item[] Plus facile a mesurer, elle r\'eduirait directement les co\^uts et l'achalandage du r\'eseau et encouragerait d'avantage de citoyens a opter pour le transport en commun;
    \item La ligne d\'econgestionne les stations achaland\'ees et introduit une nouvelle client\`ele (elle va loin):
    \item[] Selon ce crit\`ere la ligne rose semble, \`a priori, excellente.
\end{enumerate}

\bigskip

\hrule height0.3pt

\bigskip

\noindent{\bfseries 2. (a)} Mes crit\`eres de d\'ecision sont 
\begin{enumerate}
    \item Si l'utilisateur va fr\'equemment en r\'egion;
    \item S'il est un conducteur fr\'equent ou est d\'ej\`a propri\'etaire;
    \item Si l'utilisateur n'habite pas Montr\'eal ou s'il est hors/loin des zones de service;
    \item Si la situation de l'utilisateur permet de partager un v\'ehicule acheter.
\end{enumerate}

\medskip

\noindent{\bfseries \phantom{2. }(b)} Les changements et nouveaux crit\`eres sont : 
\begin{enumerate}
    \item De fa\c con analogue pour le (1), un cycliste qui planifie faire du v\'elo-tourisme
        choisira probablement d'acheter son propre v\'elo;
    \item Les points (2) et (3) semblent tout aussi pertinents;
    \item Le v\'elo est une activit\'e d'avantage saisonni\`ere
        et un utilisateur poura\^it ne pas vouloir entreposer son propre v\'elo 4 mois par ann\'ee.
\end{enumerate}


\end{document}

