\documentclass[12pt,letterpaper]{article}
\usepackage[french]{babel} 
\usepackage[T1]{fontenc}
\usepackage{lmodern}
\usepackage[utf8]{inputenc}
\usepackage{amsmath}
\usepackage[margin=3cm]{geometry}
\usepackage{graphicx}
\usepackage{url}
\usepackage{hyperref}
\usepackage{array}
\newcolumntype{L}[1]{>{\raggedright\let\newline\\\arraybackslash\hspace{0pt}}m{#1}}
\newcolumntype{R}[1]{>{\raggedleft\let\newline\\\arraybackslash\hspace{0pt}}m{#1}}

\begin{document}
% \section*{Validité du modèle:}

% Nous croyons que le modèle colle à la réalité sous plusieurs aspects. Lorsque la densité
% de circulation est faible les véhicules semblent en mesure d'atteindre une vitesse de
% croisière près de leur vitesse maximale. La voie la plus a gauche maintient généralement
% une vitesse supérieure aux autres puisqu'elle n'est pas exposée à la vitesse maximale des
% semi-remorques. De plus, les vitesses moyennes de chaque voie diminuent à mesure que la
% densité augmente ce qui est le comportement attendu.

% Nous n'avons pas accès à des données routières réelles qui nous aurait permis de faire une
% analyse contre notre simulation. Nous sommes toutefois en mesure de le comparer à un
% modèle de référence qui nous à été fournis. Qualitativement, on peut tout de suite
% constater que notre modèle n'adopte pas l'effet typique d'un bouchon de circulation. Nous
% croyons que l'ajout de conducteur distrait aurait un impact positif en ce sens.

% Finalement, il nous incombe de discuter des colisions. Notre modèle était initialement
% configuré pour qu'un véhicule freine suffisament pour éviter d'admettre le véhicule en
% avant dans sa zone de sécurité. Ce comportement engendrait des freinages brusque et
% puisque nos conducteurs ne vérifie que l'état du véhicule immédiatement devant eux un
% effet de chaine engendrait fréquemment des accidents. Après avoir relaxé le mécanisme de
% freinage notre modèle fut en mesure de fonctionner gracieusement à une augmentation de la
% densité et de la durée de la simulation. Le modèle 1 était sérieusement affecté par ce
% problème et nous pensons que la règle conventionnelle de revenir à la voie de départ
% engendre tout simplement plus de changements de voies ce qui augmente les risques de
% collisions. Il serait pertinent de comparer le nombre total de changements de voies dans
% chacun des deux modèles.

\section{Conclusion}

Nous avons tenté de construire un modèle pour évaluer la performance d'un système muni de la règle
proposée contre un autre permettant au conducteur de conserver la voie à l'extrême gauche et pour
revenir à la voie centrale seulement lorsqu'il en tire profit. Notre modèle semble indiquer que la
vitesse moyenne et le débit sont supérieurs lorsque la règle proposée est adoptée. Par contre, sous
cette condition, le nombre d'accidents enregistré était supérieur sous une densité de circulation
très élevée.

Notre modèle semble nous apprendre autre chose. D'abord que toute recommendation de conduite qui
peut engendrer une augmentation du nombre de changements de voie devra être évaluée sérieusement
avant d'être adoptée. Nous pensons aussi que les conducteurs, quoiqu'il peuvent être distrait, sont
apte à ajuster leur vitesse de façon sécuritaire durant l'heure de pointe. Nous croyons toutefois
que les conducteurs ont une mauvaise capacité à conduire d'une façon à minimiser leur impact sur
l'heure de pointe. En particulier, un contrat social qui engagerait les conducteurs à ne pas trop
déroger d'une vitesse idéale commune et réduire le nombre de changements de voie (souvent
inutiles) ferait des miracles pour des conducteurs pressés d'arriver à la maison.

Il est clair à nos yeux que l'élément de distraction stochastique (un véhicule qui ralentit
aléatoirement) aurait été essentiel au modèle. Nous pensons que ce dernier aurait introduit la vague
de traffic\cite{wave} familière à laquelle on s'attendait. Il est possible que nos deux modèles
auraient eu des comportements différentes, ou encore plus prononcés, sous de telles conditions.

Au final, nous croyons que le résultat du modèle est crédible et il suffirait de relaxer la règle
énoncée lors de période de gros traffic pour éliminer le nombre irréaliste d'accrochage observé sur
une période de 15 minutes.

\begin{thebibliography}{9}
    \bibitem{wave} \href{https://en.wikipedia.org/wiki/Traffic_wave}{Traffic wave}
\end{thebibliography}

\end{document} 
