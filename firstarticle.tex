% First document, firstarticle.tex

\documentclass{amsart}
\usepackage{graphicx}

\newcommand{\pdelta}{\pmod{\delta}}
\DeclareMathOperator{\length}{length}

\newtheorem{theorem}{Theorem}

\begin{document}
\title{A technical result\\ for congruences of finite lattices}
\author{G. Gr\"atzer}
\address{
    Department of Mathematics\\
    University of Manitoba\\
    Winnipeg, MB R3T 2N2\\
    Canada
}
\email[G. Gr\"atzer]{gratzer@me.com}
\urladdr[G. Gr\"atzer]{http://tinyurl.com/gratzerhomepage}
\date{March 21, 2015}
\subjclass[2010]{Primary: 06B10.}
\keywords{finite lattice, congruence.}
\maketitle

\begin{abstract}
    We present a technical result for congruences on finite lattices.
\end{abstract}

\begin{figure}[hbt]
    \includegraphics{covers}
    \caption{Theorem~\ref{T:technical} illustrated}
    \label{F:Theorem}
\end{figure}

\clearpage

$x \equiv y \pmod{\delta}$+

$x \equiv y \pdelta$

\begin{equation}
    x = 2 + 39 \label{T:technical}
\end{equation}

\clearpage

length\,$u$, $\length u$, $length u$

\end{document}
