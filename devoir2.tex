\documentclass[12pt,leqno]{article}

\usepackage{amsmath,amsthm}
\usepackage{amsfonts}
\usepackage[psamsfonts]{amssymb}
\usepackage{palatino,euler} 
\usepackage[applemac]{inputenc}
\usepackage[french]{babel}
\usepackage[T1]{fontenc}
\usepackage{graphicx}
\usepackage{tikz}
\usepackage{hyperref}
\usepackage{xcolor}


\setlength{\topmargin}{0pt}
\setlength{\headheight}{0pt}
\setlength{\headsep}{0pt}
\setlength{\textheight}{612pt}
\setlength{\textwidth}{400pt}
\setlength{\marginparwidth}{0pt}
\setlength{\oddsidemargin}{36pt}
% \setlength{\parindent}{0pt}

\newcommand\modif{\color{purple}}


\begin{document}

{\bf \large MAT 1460 : rapport - devoir 3 - Hypoth\`eses, variables ind\'ependantes et d\'ependantes}


\medskip
{\bf Pr\'enom Nom} : David Grenier

{\bf Date} : \today

\bigskip
\bigskip

\hrule height0.3pt
\bigskip

\noindent
{\bfseries 1. Mod\`ele SIR modifi\'e pour l'ajout d'un vaccin:}\\
Voici le mod\`ele obtenu avec les {\modif modifications}:
\begin{align*}
    \modif \mathcal V &\modif = \nu\varepsilon P\\
    S(t) &= S(t-1) - \kappa {\modif(}S(t-1){\modif -\mathcal V)}I(t-1)\\
    I(t) &= I(t-1) + \kappa {\modif(}S(t-1){\modif -\mathcal V)}I(t-1) - \gamma I(t-1)\\
    R(t) &= R(t-1) + \gamma I(t-1)\\
    \text{o\`u } P &= S(t) + I(t) + R(t) \text{\quad({\modif P inchang\'e par l'introduction de $\mathcal V$})}
\end{align*}

\begin{table}[h]
    \centering
    \begin{tabular}{|l|c|l|}\hline
        Nom &V/P &Description\\\hline
        $\kappa$ &P &taux d'infectiosit\'e\\\hline
        $\gamma$ &P &taux de r\'ecup\'eration\\\hline
        $P$ &P &population totale\\\hline
        \modif$\boldsymbol\nu$ &\modif P &\modif proportion de la population vaccin\'ee\\\hline
        \modif$\varepsilon$ &\modif P &\modif taux d'efficacit\'e du vaccin\\\hline
        $t$ &V (entr\'ee) &temps\\\hline
        \modif$\mathcal V$ &\modif V &\modif \# individus immunis\'es\\\hline
        $S(t)$ &V (sortie) &\# d'individus susceptibles\\\hline
        $I(t)$ &V (sortie) &\# d'individus infect\'es\\\hline
        $R(t)$ &V (sortie) &\# d'individus qui ont r\'ecup\'er\'es\\\hline
    \end{tabular}
\end{table}

\noindent
Le mod\`ele est sujet aux hypoth\`eses suivantes:
\begin{enumerate}
    \item $I(0) > 0$ et $S(0) = P-I(0)$, le d\'ebut de l'\'epid\'emie;
    \item La population est finie et positive;
    \item $\kappa$ est suffisamment petit pour \'eviter que $S(t) < 0$ et $I(t) > P$;
    \item Un individu ayant r\'ecup\'er\'e ne pourra pas \^etre infect\'e de nouveau;
    \modif\item Le vaccin aura \'et\'e donn\'e avant le d\'ebut de l'\'epid\'emie;
    \modif\item Un vaccin\'e dont le vaccin est efficace pr\'eserve son immunit\'e ind\'efiniment.
\end{enumerate}

\clearpage

% \hrule height0.3pt
% \bigskip

\noindent
{\bfseries 2. Mod\`ele d'autoroute modifi\'e pour l'ajout d'une voie:}\\
Voici le mod\`ele obtenu avec les {\modif modifications}:
\begin{alignat}{3}
    v_{\modif V}(i) &\to \min(v_{\modif V}(i)+1,v_{\max})\label{AC}\\
    v_{\modif V}(i) &\to \min({\modif d_V(x_V(i))}\footnotemark, v_{\modif V}(i))\label{DC}\\
    v_{\modif V}(i) &\to \max(v_{\modif V}(i-1),0) &{}&\text{(avec probabilit\'e $p$)}\label{DT}\\
    x_{\modif V}(i) &\to x_{\modif V}(i) + v_{\modif V}(i)\label{AV}\\
    &\modif x_{V \to V\oplus 1}(i)
        &{}&\modif\text{(si $d_{V \oplus 1}(x_V(i)) > d_V(x_V(i))$)}\label{CV}
\end{alignat}\footnotetext{La s\'emantique de $d()$ a changer pour trouver le prochain v\'ehicule \`a partir de l'emplacement plut\^ot que le num\'ero de s\'equence sachant que le v\'ehicule $i$ de la voie 0 ne restera pas \`a c\^ot\'e du v\'ehicule $i$ de la voie 1.}
Les \'etapes sont \eqref{AC} acc\'el\'eration,
\eqref{DC} d\'ec\'el\'eration,
\eqref{DT} distraction,
\eqref{AV} on avance les v\'ehicules selon leur vitesse et
\eqref{CV} on change les v\'ehicules pouvant profiter d'une opportunit\'e. On effectue chaque \'etape pour chaque voie avant de passer \`a l\'etape suivante.

Puisqu'un v\'ehicule ne change de voie que s'il y a plus d'espace devant lui apr\`es que tous les
v\'ehicules se soient d\'eplacer il n'y aura pas de collisions.

\begin{table}[h]
    \centering
    \begin{tabular}{|l|c|l|}\hline
        Nom &V/P &Description\\\hline
        $v_{\max}$ &P &La vitesse maximale\\\hline
        $p$ &P &La probabilit\'e $p$ qu'un conducteur soit distrait\\\hline
        $x_{\modif 0},v_{\modif 0},x_{\modif 1},v_{\modif 1}$ &V (entr\'ee) &L'\'etat initial du syst\`eme\\\hline
        $v_{\modif V}(i)$ &V &La vitesse du v\'ehicule $i$ {\modif dans la voie $V$}\\\hline
        \modif $d_V(i)$ &V\modif  &\modif Distance au prochain v\'ehicule de la voie $V$ de la case i\\\hline
        \modif $V$ &\modif V &\modif Variable discrete prenant le \# de la voie, soit 0 o\`u 1\footnotemark\\\hline
        $x_{\modif 0},x_{\modif 1}$ &V (sortie) &Les positions durant/apr\`es la simulation\\\hline
    \end{tabular}
\end{table}
\footnotetext{On utilise l'addition modulo 2 pour obtenir l'autre voie: $1 \oplus 1 = 0, 0 \oplus 1 = 1$}

\noindent
Le mod\`ele est sujet aux hypoth\`eses suivantes:
\begin{enumerate}
    \item On assume que $(+)$ reviens au d\'ebut lorsqu'on est \`a la t\^ete;
    \item L'acc\'el\'eration maximale est de 1 unit\'e;
    \item On v\'ehicule peut freiner a 0 \`a l'int\'erieur d'une unit\'e de temps;
    \modif\item La longueur discr\'etis\'ee de l'autoroute est suffisante pour accomoder tous les v\'ehicules sur une seule voie;
    \modif\item On change de voie uniquement pour avoir plus d'espace libre et \'eventuellement acc\'el\'erer;
    \modif\item (d\'etail d'impl\'ementation) Un v\'ehicule qui change de voie occasionne la r\'eindexation du num\'ero des v\'ehicules.
\end{enumerate}

\end{document}
