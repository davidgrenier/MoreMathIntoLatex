\documentclass{sample}
\newcommand{\bsl}{\texttt{\symbol{92}}}
\begin{document}
    `subdirectory irreducible' and ``subdirectory irreducible''

    ``she replied, `No.'\,''

    Mean-Value Theorem, see pages~23--45 in Schmidt--Freid
    and the full name P.~Neukomm

    A long dash---called an \emph{em dash}---is used

    \begin{align}
        \phi(x) = 2^x\label{T:main}\\
        \zeta(x) = x^2\label{S:intro}
    \end{align}

    Theorem~\ref{T:main} in Section~\pageref{S:intro}. Donald~E. Knuth.  
    assume that $f(x)$ is (a)~continuous, (b)~bounded the lattice~$L$.
    Peter~G.~Neukomm.

    Test \textbackslash\ and \texttt{\textbackslash} and \bsl\ and \texttt{\bsl}

    The | (vertical bar) shouldn't be used
    instead use \bsl\texttt{textbar} (\textbar) in text
    and \texttt{\bsl mid} in formulae as in $\{ x \mid x > 3\}$

    The \^\ is invalid by itself, you must type in \bsl\texttt{\^}.
    Obviously, so is the \$ sign, which requires \bsl\texttt{\$}.
    \{ and \}, beign used in command arguments also require to
    be prefixed with \bsl. The \#, \% and \~\ all require \bsl.
    However \texttt{@} typesets as @.

    The * is high up on the line in text, however centered in formula
    such as $2 * 3$. You can get a centered * with \bsl\texttt{textasteriskcentered}.
    Such as \textasteriskcentered.

    You should never use \texttt{"}, perhaps except in code:

    \texttt{print("Hello!")}

    While the \bsl\texttt{"} will create the umlaut: Gr\"atzer.
\end{document}
