\documentclass{amsart}


\setlength\parindent{0pt}
\begin{document}
Qu'est-ce que la th\'eorie de l\'elimination nous permet de r\'esoudre?

$n$ \'equations quadratiques \`a n inconnues:
\[ \sum_{1\leq i\leq j\leq n} b_{i,j,k}x_ix_j + \sum_{1\leq i\leq n} c_{i,k}x_i + d_k = 0 \quad k \in \{1,\dots,n\} \]

\vspace{10pt}
Qu'est-ce qu'un syst\`eme d'\'equation r\'esoluble?

Si la solution g\'en\'erale s'exprime avec les seules op\'erations: $+, -, \times, \div, \sqrt[n]{}$

\vspace{10pt}
Quel est le cas g\'en\'eral pour la solution de syst\`emes d'\'equations?

Pour $f \colon \mathbb R^n \to \mathbb R^m \quad n \geq m$ o\`u $m$ est le nombre d'\'equations.
\[ \vec f(\vec x) = \vec 0 \]

\vspace{10pt}
Qu'est-ce qu'une d\'eriv\'ee partielle?

Pour $f \colon \mathbb R^n \to \mathbb R$, la d\'eriv\'ee partielle de $f$ par rapport \`a $x_i,\,(i \leq i \leq n)$ au point $\vec x $ est:
\[ \lim_{h\to 0} = \frac{f(\vec x + h\cdot \vec e_i) - f(\vec x)}h \]

Qu'est-ce que la classe $\mathcal C^1$ si $\dfrac{\partial f}{\partial x_i}, 1 \leq i \leq n$ sont toutes des fonctions continues.

Si $\dfrac{\partial f}{\partial x_i}$ existe et est continue, on peut r\'eit\'er\'e avec $\dfrac{\partial}{\partial x_j}\dfrac{\partial f}{\partial x_i}$. Si cette derni\'ere
existe est est continue, on dit qu'elle est de classe $\mathcal C^2$.

\vspace{10pt}
Qu'est-ce que $f$ de classe $\mathcal C^\infty$?

\vspace{10pt}
Quelles remarques peut-on faire dans $\mathcal C^k$?
\begin{enumerate}
    \item Si $f$ est de classe $\mathcal C^k$, l'ordre de d\'erivation n'a pas d'importance;
    \item On peut remplacer $\mathbb R^n$ par n'importe quel ouvert de $\mathbb R^n$;
\end{enumerate}

\vspace{10pt}
Quelle est la distinction pour $\vec f \colon \mathbb R^n \to \mathbb R^m$?
\[ \frac{\partial \vec f}{\partial x_i} = \left( \frac{\partial f_1}{\partial x_i}, \dots, \frac{\partial f_n}{\partial x_i} \right) \]

\vspace{10pt}
Qu'est-ce qu'une fonction $\vec f \colon \mathbb R^n\to \mathbb R^m$ dite analytique?
S'il existe une \'equation de la forme:
\[ \sum_{k=0}^\infty \sum_{i_1\leq i_2\leq \dots \leq i_k} a_{i_1i_2\dots i_k}^k x_{i_1}\dots x_{i_k} \]
Qui converge vers la valeur de $\vec f(\vec x)$ pour tout $\vec x$.

\vspace{10pt}
Quel th\'eor\`eme a-t-on pour une fonction analytique?
\[ f \text{ est } \mathcal C^\infty \]
Et ses d\'eriv\'ees de tout order se calculent en d\'erivant l'expression terme \`a terme.

\vspace{10pt}
Qu'est-ce que le th\'eor\`eme des fonctions implicites?

Soit $\vec f\colon \mathbb R^n \to \mathbb R^m,\, n \geq m,\, \mathcal C^k$ pour $k \geq 1$, pour $J_{\vec f}$ le Jacobien de $\vec f$:\\
Supposons que pour un $\vec x$ il existe une matrice $m\times m$ de $J_{\vec f}$ (un sous-ensemble des colonnes de $J_{\vec f}$) tel que cette derni\`ere soit de 
d\'eterminant non-nul. Alors le sous-ensemble $S = \vec f^{-1}(\vec f(\vec x))$ est un sous-espace de $\mathbb R^n$ qui est lisse et de dimension $n-m$ pr\`es de $\vec x$.

\vspace{10pt}
Qu'est-ce qui est particulier lorsque $n=m$?

L'ensemble solution de $f^{-1}(y)$ sera form\'e de points isol\'es et donc c'est impossible.

\vspace{10pt}
Donner un exemple suivant le th\'eor\`eme des fonctions implicite.

\end{document}
