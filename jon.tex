% 7.50
\documentclass[executivepaper,leqno]{amsart}
\usepackage[makeroom]{cancel}
\usepackage[margin=2.5cm]{geometry}
\usepackage{graphicx}
\usepackage{amsmath,amsthm,esint}
\usepackage{amssymb}
\usepackage{hhline}
\usepackage{lineno}
\usepackage{siunitx}
\usepackage{mathtools}
\usepackage{tikz}
\usepackage[super]{nth}
\usepackage{centernot}
\usepackage{stackengine}
\usepackage{dcolumn}
\usepackage{multicol}
\usepackage{bbm}

\newcolumntype{2}{D{.}{}{2.0}}
% \setlength\parskip{1\baselineskip}

\def\Mline#1{\hspace*{-\leftmargin}\parbox{\textwidth}{\begin{align*}#1\end{align*}}}
\NewDocumentCommand\overarrow{O{=} O{\uparrow} m}{%
    \overset{\makebox[0pt]{\begin{tabular}{@{}c@{}}\ensuremath{#3}\\[0pt]\ensuremath{#2}\end{tabular}}}{#1}
}
\newcommand\nunlhd{\mathrel{\ooalign{$\lneq$\cr\raise.22ex\hbox{$\lhd$}\cr}}}
\newcommand\rring[1]{{\mathop{\kern0pt #1}\limits^{\vbox to-1.85ex{\kern-2ex\hbox to 0pt{\hss\normalfont\kern.1em \r{}\kern-.45em \r{}\hss}\vss}}}}
\newcommand\opn[1]{\operatorname{#1}}
\newcommand\var[1]{\opn{Var}\left(#1\right)}
\newcommand\esp[1]{\opn{\mathbb E}{\left[#1\right]}}
\newcommand\esps[1]{\opn E^2{\left[#1\right]}}
\newcommand\espp[1]{\opn E_p{\left[#1\right]}}
\newcommand\cov[2]{\opn{Cov}\left(#1,#2\right)}
\newcommand\prob[1]{\opn{P}\!\left[#1\right]}
\newcommand\indep[2]{#1\bot\!\!\!\bot #2}
\newcommand\ol[1]{\overline{#1}}
\renewcommand\gcd[2]{\left(#1,#2\right)}
\newcommand\llb[1]{\mathbbm{#1}}
\newcommand{\bigslant}[2]{{\raisebox{.2em}{$#1$}\!\!\left/\!\!\raisebox{-.2em}{$#2$}\right.}}
\newcommand\im[1]{\opn{Im}\left(#1\right)}
\newcommand\surj\twoheadrightarrow
\newcommand\inj{\xhookrightarrow{}}
\newcommand\id[1]{\llb 1_{\!\textsc{#1}}}
\newcommand\toinf{\overset{n\to\infty}\longrightarrow}
\renewcommand\deg[1]{\opn{deg}(#1)}
\newcommand\del[2]{\frac{\partial#1}{\partial#2}}
\newcommand\pp[1]{\frac{\partial}{\partial#1}}
\newcommand\ppp[1]{\frac{\partial^2}{\partial{#1}^2}}
\renewcommand\neg[1]{\sim\!\!#1}

\begin{document}
% J'ai travaill\'e pour JaPache Web Solutions sous la supervision de Jonathan Warmington de
% Janvier \`a D\'ecembre 2016. Du 13 Janvier au 11 Ao\^ut 2016 j'ai particip\'e au
% d\'eveloppement de la plateforme Gradesavers tel que l'historique du code peut le
% confirmer.

% Mon exp\'epience en d\'eveloppement de logiciel est principalement \`a l'arri\`ere plan
% (r\^egle d'affaires) mais aussi du c\^ot\'e des pages web ce qui s'est refl\'et\'e dans
% mes contributions au projet Gradesavers. J'ai fait l'entretien et le d\'eveloppement
% de fonctionnalit\'es, impl\'ementer des r\`egles d'affaires sophistiqu\'es et connect\'es
% l'interface web \`a l'API.

(page 2 de 2)\\
Malgr\'e mon exp\'erience dans le d\'eveloppement logiciels, Jon Warmington
\'etant plus familier avec la plateforme que moi, \'etait un d\'eveloppeur plus productif.
J'ai \'et\'e t\'emoins \`a plusieurs reprises que les heures factur\'es \'etaient
inf\'erieures aux heures travaill\'ees pour compenser \`a cette diff\'erence en
productivit\'e.

\`A ma connaissance, ma participation au projet n'a jamais fait l'objet de reproches de la
part du client et de nouvelles fonctionnalit\'es \'etaient demand\'es sur une base
r\'eguli\`ere. En particulier, j'ai visiter le site web actuel du client (comme toute
personne avec un minimum d'exp\'erience est en mesure de le faire) et ai \'et\'e t\'emoin
que plusieurs pages contiennent toujours mon nom et celui de Jonathan. De plus, les appels
de service trahissent l'utilisation de l'API que nous avions impl\'ementer. En inspectant
le traffic web, la familiarit\'e des requ\^etes me rappelles des souvenirs de plusieurs
sections de la base de code. Ce dernier est sans aucuns doutes encore en usage
aujourd'hui.

Au cours de ma participation \`a ce projet, nous n'avons jamais fait patienter le client
significativement pour r\'egler des probl\`emes. Je n'ai par ailleurs jamais \'et\'e
t\'emoin que Jonathan ait pris des d\'ecisions qui nous aurait emp\^echer de rencontrer
les attentes du clients.
\end{document}
