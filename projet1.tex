\documentclass[12pt]{article}

\usepackage{amsmath,amsthm}
\usepackage{amsfonts}
\usepackage[psamsfonts]{amssymb}
\usepackage{palatino,euler} 
\usepackage[applemac]{inputenc}
\usepackage[french]{babel}
\usepackage[T1]{fontenc}
\usepackage{graphicx}
% \usepackage{tikz}
\usepackage{hyperref}


\setlength{\topmargin}{0pt}
\setlength{\headheight}{0pt}
\setlength{\headsep}{0pt}
\setlength{\textheight}{612pt}
\setlength{\textwidth}{400pt}
\setlength{\marginparwidth}{0pt}
\setlength{\oddsidemargin}{36pt}
\setlength{\parskip}{\baselineskip}
% \setlength{\parindent}{0pt}

\newcommand\rn[1]{\romannumeral #1}


% \title{MAT1460\\[3ex]L'\'epid\'emie d'influenza 2017-2018 demeure-t-elle dangereuse?}
% \author{Pelletier, Xavier\\[1ex] Chen, Siying\\[1ex]L\'evesque, \'Etienne\\[1ex] Grenier, David}
% \date{}
\begin{document}

\begin{enumerate}
    \item Trouver la r\'ef\'erence pour les d\'epassements par la droite;
\end{enumerate}

\section{Types de v\'ehicules}
\begin{table}[h]
    \begin{tabular}{|l|l|l|l|l|l|}\hline
        Type &Proportion &Longueur(m) &Masse (kg) &Accel\'eration($m/s^2$) &$V_{\max}$(m/s)\\\hline
        Camions &0.25 &14.6 &9750 &0.6 &29\\\hline
        Voitures &0.73 &4.5 &1850 &3 &32.5\\\hline
        Motocyclette &0.02 &2.2 &175 &7 &32.5\\\hline
    \end{tabular}
\end{table}

\section{Crit\`eres}
\begin{enumerate}
    \item D\'ebit avec pond\'eration de chaque voies
    \item[] (Moyenne g\'eom\'etrique ou \'ecart-type, etc);
    \item Minimiser la somme des \'energies des collisions;
    \item[] $\frac{\max(m_1,m_2)}2(v_1-v_2)^2$;
    % \item[] (Temps avant la premi\`ere collision, diff\'erence de vitesse, etc);
    % \item Le moins de variations de vitesse possible;
    % \item Nombre de d\'epassements totals;
    % \item Les types de d\'epassements (gauche ou droite);
    % \item Formule pour mesurer le graphe des bouchons;
    % \item[] Densit\'e uniforme;
\end{enumerate}

\section{Hypoth\`eses}
\begin{enumerate}
    \item Les collisions sont in\'elastique;
    \item Pas de rampes gauches;
    \item Syst\`eme isol\'e des conditions ext\'erieures;
    \item Autoroute seulement ou routes avec feux/stops;
    \item Lorsqu'un v\'ehicule acc\'el\`ere, il va au max;
    \item Autoroute est une boucle;
\end{enumerate}

\section{Param\`etres}
\begin{enumerate}
    \item $\rho$: Taux d'ajout et de retrait de v\'ehicules;
    \item Vitesse des v\'ehicules entrant/sortant;
    \item \'Etat initial du syst\'eme (positions/vitesse de chaque v\'ehicule);
    \item O\`u et quand de nouveaux v\'ehicules s'ajoutent et \`a quelle vitesse;
\end{enumerate}

\section{Conditions initiales}
\begin{enumerate}
    \item Pas de collisions (pas d'overlap) + vitesse initiale entre 50 et $V_{\max}$;
    \item Nombres de v\'ehicules est, par voie: $\frac14,\frac12,\frac14$;
    \item L'autoroute a 5km;
    \item Un nombre fixe de v\'ehicule;
\end{enumerate}

\section{R\`egle}
Mod\`ele de r\'ef\'erence:
\begin{enumerate}
    \item Un v\'ehicule ne changera jamais de voie s'il peut acc\'el\'erer dans la sienne;
    \item Un v\'ehicule veut toujours acc\'el\'erer;
    \item $b\to a\colon$ est pour dispara\^itre \'eventuellement avec une probabilit\'e $\rho$;
    \item $b\to c \colon dist(x_{i+1}(b)) < dist(x_{i+1}(c))$;
    \item $c \to b \colon$ possible de le faire sans d\'ec\'el\'erer;
\end{enumerate}

\end{document}
