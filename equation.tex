\documentclass[leqno]{sample}
\numberwithin{equation}{section}
\usepackage{amsmath}

\begin{document}

\begin{equation}
    \label{E:firstIntegral}
    \int_0^{\pi} \sin x\,dx = 2
\end{equation}

\begin{equation}
    \tag{Int}
    \int_0^{\pi} \sin x\, dx = 2
\end{equation}

\begin{align}
    r^2 &= s^2 + t^2, \label{E:Pyth}\\
    2u + 1 &= v + w^{\alpha}, \label{E:alpha}\\
    x &= \frac{y+z}{\sqrt{s+2u}}; \label{E:frac}
\end{align}

\begin{align}
    h(x) &= \int \left(
                    \frac{f(x) + g(x)}{1+f^2(x)} +
                    \frac{1 + f(x)g(x)}{\sqrt{1 - \sin x}}
                        \right)\,dx\\
    &= \int \frac{1 + f(x)}{1 + g(x)}\,dx - 2 \tan^{-1}(x-2) \notag
\end{align}

\clearpage

See~\eqref{E:firstIntegral} on page~\pageref{E:firstIntegral}

\begin{align}
    x &= x \wedge (y \vee z) && \text{(by distributivity)}\\
    &= (x \wedge y) \vee (x \wedge z) && \text{(by condition (M))} \notag\\
    &= y \vee z \notag
\end{align}

\[
    f(x) =
    \begin{cases}
        -x^2,       &\text{if $x < 0$;}\\
        \alpha + x, &\text{if $0 \leq x \leq 1$;}\\
        x^2,        &\text{otherwise.}
    \end{cases}
\]

\end{document}
