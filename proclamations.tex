\documentclass{amsart}

\newtheorem{definition}{Definition}
\theoremstyle{definition}
\newtheorem{lemma}{Lemma}[section]
\swapnumbers
\newtheorem{proposition}[lemma]{Proposition}
\theoremstyle{remark}
\newtheorem*{note}{Note}

\begin{document}
\begin{definition}[The Fuchs-Schmidt Theorem]
    The statement of the theorem
\end{definition}

\begin{definition}\label{D:prime}
    \hfill
    \begin{enumerate}
        \item $u$ is \emph{bold} if $u = x^2$.\label{mi1}
        \item $u$ is \emph{thin} if $u = \sqrt{x}$.\label{mi2}
    \end{enumerate}
\end{definition}

\begin{lemma}
    This is a fun topic
\end{lemma}

\section{}

\begin{lemma}
    Let us restate: this is a fun topic
\end{lemma}

\begin{proposition}\label{P:fun}
    This is a much more fun topic
\end{proposition}

\begin{note}
    Some people fuck things up for others
\end{note}

\begin{proof}[Proof of Proposition~\ref{P:fun}]
    \hfill
    \begin{description}
        \item[$\alpha$] A little
        \item[$\beta$] quick brown
    \end{description}

    This is a proof, delimited by the q.e.d.\ symbol.
\end{proof}

\begin{proof}
    Now the proof follows from the equation
    \[
        a^2 = b^2 + c^2.\qedhere
    \]
\end{proof}
\end{document}
